\usepackage{graphicx} 
\usepackage{amsmath}
\usepackage{amssymb}
\usepackage{authblk}
\usepackage[skip=10pt plus3pt]{parskip}
\usepackage{csquotes}
\usepackage{url}
\usepackage{mleftright}
\usepackage{float}
\usepackage{enumitem}
\usepackage[authoryear]{natbib}
\usepackage[a4paper, margin=1in]{geometry}
\usepackage{tcolorbox}
\usepackage{enumitem}
\usepackage{caption}
\usepackage{endnotes}
\usepackage{chngcntr}
\usepackage[colorlinks=true, linkcolor=black, urlcolor=black, citecolor=black]{hyperref}


% packages required for tables generated with kable()
\usepackage{booktabs}
\usepackage{longtable}
\usepackage{array}
\usepackage{multirow}
\usepackage{wrapfig}
\usepackage{float}
\usepackage{colortbl}
\usepackage{pdflscape}
\usepackage{tabu}
\usepackage{threeparttable}
\usepackage{threeparttablex}
\usepackage[normalem]{ulem}
\usepackage{makecell}
\usepackage{xcolor}
\usepackage{setspace}

\usepackage{pifont}  % For checkmark and cross symbols
\usepackage{lmodern} % For a smoother font

% Define checkmark and cross commands with crayon-like style
\newcommand{\cmark}{\ding{51}} % Checkmark symbol from pifont
\newcommand{\xmark}{\ding{55}} % Cross symbol from 
\usepackage{verbatim}

% Increase the font size of the endnotes
\renewcommand{\enoteformat}{\normalsize\noindent\theenmark. \setlength{\parindent}{0pt}}


\setcounter{equation}{0}
\renewcommand{\theequation}{\arabic{equation}}
\DeclareMathOperator*{\var}{Var}
\DeclareMathOperator{\E}{\mathbb{E}}
\DeclareMathOperator*{\cov}{Cov}
\DeclareMathOperator*{\se}{se}
\DeclareMathOperator*{\corr}{Corr}
\newcommand{\reals}{\mathbb{R}}
\DeclareMathOperator*{\argmax}{arg\,max}
\DeclareMathOperator*{\argmin}{arg\,min}
\newcommand{\PPI}{\mathrm{PP}}

\newcommand{\mytitle}{The Mixed Subjects Design: \\ Treating Large Language Models as (Potentially) Informative Observations}

\newcommand{\mydate}{ 
}

\newcommand{\myabstract}[1]{%
\textbf{Abstract:} Large Language Models (LLMs) promise to transform the social sciences through cost-effective predictions of human behavior. However, despite growing evidence that LLMs can misrepresent such behavior, current approaches to studying causal effects with LLMs require researchers to assume that predicted and observed behavior are \textit{interchangeable}. Instead, we argue that human subjects should serve as a gold standard to correct misrepresentations within a \textit{mixed subjects design}. This paradigm offers valid and more precise estimates of causal effects at a lower cost than experiments relying solely on human subjects. We demonstrate---and extend---prediction-powered inference, a statistical method that instantiates the mixed subjects design. Our innovation is a power analysis for optimally choosing between \textit{informative but costly} human subjects and \textit{less informative but cheap} predictions of human behavior.
Mixed subjects designs could enhance scientific productivity and reduce inequality in access to costly evidence on research questions by offering valid, precise, and cost-effective inferences on causal effects and other parameters.
%Empirically, we show that the mixed subjects design reproduces the statistical precision and reproduces treatment effects of a complex experiment---even though the predicted behavior on its own does not. 
\\ \\
\textbf{Keywords:} Mixed Subjects Design, Prediction-Powered Inference, PPI Correlation, Experiments, Power Analysis, Machine Learning, Large Language Models, Moral Machine experiment, Computational Social Science
}
